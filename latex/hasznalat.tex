\documentclass{article}
\usepackage[a4paper, total={18cm, 24cm}]{geometry}
\usepackage[magyar]{babel}
\usepackage[utf8]{inputenc}
\usepackage{graphicx}
\usepackage{t1enc}
\usepackage{amsmath}
\usepackage{epstopdf}
\usepackage{hyperref}
\usepackage{gensymb}


\newenvironment{rcases}
  {\left.\begin{aligned}}
  {\end{aligned}\right\rbrace}

\title{Vezetők terének számítása Biot-Savart törvénnyel és vektormező kirajzolása}
\author{Pongó Tivadar}
\date{}
\begin{document}
\maketitle
\noindent
\textbf{Dátum:} 2015. Május \\
\textbf{E-mail cím:} pongo-t@vipmail.hu

\section*{Használat}
\subsection*{Gombok}
Jobbra, balra nyíl: vezetőelrendezés választása (szolenoid, toroid, egyenes) \\
Fel, le nyíl: vektormező választása (sík, térbeli) \\
W: vektormező eltolása fel \\
S: vektormező eltolása le \\
A: vektormező eltolása balra \\
D: vektormező eltolása jobbra \\
Q: vektormező eltolása előre \\
E: vektormező eltolása hátra \\
R: nyilak hosszának növelése \\
F: nyilak hosszának csökkentése \\
+: vektormező pontjai közti távolság növelése \\
-: vektormező pontjai közti távolság csökkentése \\
egérgörgő: kamera és a középpont közti távolság növelése/csökkentése \\
bal egérgomb + egérmozgatás: kamera helyzetének változtatása

\subsection*{Új vezetők hozzáadása}
A programozói dokumentáció alapján forráskód (kicsiny) átírásával vagy a készítőtől lehet kérni.

\subsection*{Új vektormező hozzáadása}
Kb. mint a fentebbi pont.

\end{document}